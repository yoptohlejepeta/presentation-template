\documentclass[aspectratio=1610]{beamer}
\usepackage[T1]{fontenc}
\usetheme{wildcat}


\title{There Is No Largest Prime Number}
\date[ISPN ’80]{27th International Symposium of Prime Numbers}
\author[Euclid]{Euclid of Alexandria \texttt{euclid@alexandria.edu}}
\titlegraphic{\includegraphics[scale=0.25]{Northwestern_white.pdf}}

% % You can directly change the colors using the following macros.
% % You must redefine colors AFTER the theme is loaded.
% % For example, these provide shades of Yale Blue (#00356b)
% \definecolor{wcprimary}{RGB}{0,53,107}      % Main color
% \definecolor{wcprimary140}{RGB}{0, 34, 70}
% \definecolor{wcprimary130}{RGB}{0, 40, 80}
% \definecolor{wcprimary120}{RGB}{0, 45, 91}
% \definecolor{wcprimary110}{RGB}{0, 50, 102}
% \definecolor{wcprimary40}{RGB}{153, 174, 196}
% \definecolor{wcprimary30}{RGB}{179, 194, 211}
% \definecolor{wcprimary20}{RGB}{204, 215, 225}
% \definecolor{wcprimary10}{RGB}{230, 235, 240}

% % Now for the alerted orange (#bd5319) and example green (#5f712d)
% \definecolor{wcalerted}{RGB}{189,83,25}
% \definecolor{wcexample}{RGB}{95,113,45}




\begin{document}

\begin{frame}
\titlepage
\end{frame}


\begin{frame} 
\frametitle{There Is No Largest Prime Number} 
\framesubtitle{The proof uses \textit{reductio ad absurdum}.} 
\begin{tblock}{Theorem}
There is no largest prime number. \end{tblock} 
\begin{enumerate} 
\item<1-| alert@1> Suppose $p$ were the largest prime number. 
\item<2-> Let $q$ be the product of the first $p$ numbers. 
\item<3-> Then $q+1$ is not divisible by any of them. 
\item<4-> But $q + 1$ is greater than $1$, thus divisible by some prime
number not in the first $p$ numbers.
\end{enumerate}
\end{frame}

\begin{frame}{A longer title}
\begin{itemize}
\item one
\item two
\end{itemize}
\end{frame}

\section{Section Example}

\begin{frame}
\frametitle{Box Examples (Default)}
You can use the Beamer default blocks in the usual way:
\begin{block}{Main Block}
This is an example block
\end{block}
\begin{alertblock}{Alert Box}
This is an alert box
\end{alertblock}
\begin{exampleblock}{Example Box}
    This is an example box
\end{exampleblock}
\end{frame}

\begin{frame}{Box Examples (tcolorbox)}
    You can also use tcolorbox style blocks with the facet pattern instead of the default beamer blocks.
    \begin{tblock}{T Block Title}
        This is a tcolorbox style block.
    \end{tblock}
    \begin{talert}{T Alert Title}
        This is a tcolorbox style alert block.
    \end{talert}
    \begin{texample}{T Example Block Title}
        This is a tcolorbox style example block.
    \end{texample}
\end{frame}

\begin{frame}{Box Examples (tcolorbox) - Facet Blocks}
    There is also a special block called tfacetbox which allows you to specify the color. This only works with non-primary (not red, green, or blue) colors, as you can't shade those easily.
    \begin{tfacetbox}[nudarkyellow]{T Block Title}
        This is a tfacetbox block.
    \end{tfacetbox}
\end{frame}

% Standout frame
\standout{Standout Slide}


\end{document}